\PassOptionsToPackage{table}{xcolor}
\documentclass[]{kiesgrube}
\nofiles
\usepackage[utf8]{inputenc}
\usepackage[english]{babel}
\usepackage[T1]{fontenc}

\usepackage{graphicx}
\usepackage{tabulary}

\usepackage{url}% do not use hyperref as its links are displaced with baposter because of the font scale 

\usepackage{blindtext}% remove for production

% Select Font %%%%%%%%%%%%%%%%%%%%%%%%%%%%%%%%%%%%%%%
%\usepackage{helvet} % closest to arial
\usepackage{bookman} % has some resemblance to Futura
%%%%%%%%%%%%%%%%%%%%%%%%%%%%%%%%%%%%%%%%%%%%%%%%%%%%%
\renewcommand{\familydefault}{\sfdefault}

\newcommand{\captionfont}{\footnotesize}
\usepackage[font=small,labelfont=bf]{caption}

\begin{document}

\begin{poster}% Set grid to false for final print
{grid=false,headerheight=2.5em}{}{SNIK Tag}{}

%%%%%%%%%%%%%%%%%%%%%%%%%%%%%%%%%%%%%%%%%%%%%%%%%%%%%%%%%%%%%%%%%%%%%%%%%%%%%%
\begin{posterbox}[name=person,column=0,row=0]{Person in charge / Employee}
Winter, Höffner, Pause
\end{posterbox}
%%%%%%%%%%%%%%%%%%%%%%%%%%%%%%%%%%%%%%%%%%%%%%%%%%%%%%%%%%%%%%%%%%%%%%%%%%%%%%
\begin{posterbox}[name=progress,below=person]{Progress compared to last presentation}
%\small
\begin{tabulary}{\textwidth}{lL}
\end{tabulary}
\end{posterbox}
%%%%%%%%%%%%%%%%%%%%%%%%%%%%%%%%%%%%%%%%%%%%%%%%%%%%%%%%%%%%%%%%%%%%%%%%%%%%%%
\begin{posterbox}[name=description,column=1,row=0]{Project Description}
\small
authoring tool for annotating ontologies while writing books
\end{posterbox}
%%%%%%%%%%%%%%%%%%%%%%%%%%%%%%%%%%%%%%%%%%%%%%%%%%%%%%%%%%%%%%%%%%%%%%%%%%%%%%
\begin{posterbox}[name=open,column=1,below=description]{Open Points / Next Planned Steps}
\footnotesize
\begin{itemize}
\item usage by Ken Kretzschmar for his Bachelor thesis
\end{itemize}
\end{posterbox}
%%%%%%%%%%%%%%%%%%%%%%%%%%%%%%%%%%%%%%%%%%%%%%%%%%%%%%%%%%%%%%%%%%%%%%%%%%%%%%
\begin{posterbox}[name=risks,column=1,below=open]{Risks / Obstacles}
\begin{itemize}
\item 
\end{itemize}
\end{posterbox}
%%%%%%%%%%%%%%%%%%%%%%%%%%%%%%%%%%%%%%%%%%%%%%%%%%%%%%%%%%%%%%%%%%%%%%%%%%%%%%
\begin{posterbox}[name=escalation,column=1,below=risks]{Escalation / Decision Needs}
\footnotesize
\begin{itemize}
\item 
\end{itemize}
\end{posterbox}
%%%%%%%%%%%%%%%%%%%%%%%%%%%%%%%%%%%%%%%%%%%%%%%%%%%%%%%%%%%%%%%%%%%%%%%%%%%%%%
\begin{posterbox}[name=event,column=0,below=progress]{Event Date Forecast}
\small
\begin{tabulary}{\textwidth}{lL}
%December 5	&\\
\end{tabulary}
\end{posterbox}
%%%%%%%%%%%%%%%%%%%%%%%%%%%%%%%%%%%%%%%%%%%%%%%%%%%%%%%%%%%%%%%%%%%%%%%%%%%%%%

\tikz{\draw (image.south west) node[anchor=west]{IMISE | MIG | Konrad Höffner | \date{}};}

\end{poster}
\end{document}

