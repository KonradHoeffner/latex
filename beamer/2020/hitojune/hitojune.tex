\PassOptionsToPackage{rgb}{xcolor}
\documentclass[aspectratio=1610]{beamer}
\usepackage{tikz}
\usepackage[utf8]{inputenc}
\usepackage{booktabs}
\usepackage{tabulary}
\usepackage{amssymb}% http://ctan.org/pkg/amssymb
\usepackage{pifont}% http://ctan.org/pkg/pifont
\usepackage{csquotes}
\newcommand{\cmark}{\ding{51}}%
\newcommand{\xmark}{\ding{55}}%
\usetheme{simple}
\usecolortheme{whiteonblack}

\title{HITO Authoring Pipeline}
\subtitle{HITO Big Blue Button Meeting}
\author{Konrad Höffner \& Thomas Pause}

% Parameters: body text, frame title, image path, (optional) note
\newcommand{\imageslide}[4][]
{
%\newgeometry{margin=0cm,top=1em}
\begin{frame}[plain]{~~~~#2}
\vspace{0.2em}
\centering\includegraphics[width=1.0\textwidth,height=0.95\textheight,keepaspectratio]{#3}
\\#1
\note{#4}
\end{frame}
%\restoregeometry
}

\begin{document}
\begin{frame}
\titlepage
\end{frame}

\begin{frame}{Goal: better Data Quality}
\begin{itemize}
  \item No defined way for the preparation of Data
  \item many approaches and suboptimal solutions
  \item no one knows exactly what to do and how to manage Data
  \item $\Rightarrow$ a universal solution is needed to handle the authoring pipeline in HITO
  \item $\Rightarrow$ we can achieve much better data quality!!
\end{itemize}
\end{frame}

\begin{frame}{Goal: Lift the Confusion}
\begin{tikzpicture}
%    \foreach \i in {1,...,30} {
\foreach [count=\count] \word in {Instance Generator, Ontology, Diagram, CSV2RDF, Tarql, Turtle, Catalogue, Classified, Citation, Issue, Protégé , ?, ?, ?, ?, ?, ?, ?, ?, ?} {
      \pgfmathparse{rnd}
      \definecolor{MyColor}{hsb}{\pgfmathresult,1,1}
      \pgfmathparse{3.0*rnd+1.0}
      \node[text=MyColor,
	rotate=rand*25]
	at (10*rnd,8*rnd) {\scalebox{\pgfmathresult}{\word}};
    };
  \end{tikzpicture}
\end{frame}

\begin{frame}{HITO---The Early Days}
\begin{itemize}
  \item at the beginning, all was well
  \item Verena modelled the HITO ontology in Protégé, and also some instance data
  \item HITO was living in a single file
  \item this single file was placed in a GitHub repository~\footnote{\url{https://github.com/hitontology/ontology}}, so that we have version control (backups, history and multi user editing)
  \item Verena: Protégé, Konrad: text editor (both on the GitHub repository), Franziska: yed diagram
\begin{itemize}
  \item Konrad and Verena are always synchronous through GitHub, but Franziska's diagram is \emph{not} synchronous! \emph{Problem:} some people think HITO "is" the diagram. But "the real HITO" is the file! (at this point). This problem can be solved by clarification.
  \item small diffs can be achieved by Konrad by editing in the Protégé way
\end{itemize}
  \item problems are solved, everything is great in ontology land!
  \item HITO is uploaded to a SPARQL endpoint, but it is not the master! Changes are not persistent there.
\end{itemize}
\end{frame}

\begin{frame}{HITO---Adolescence}
\begin{itemize}
  \item Problems are solved, everything is great in one file land?
  \item We don't only want ontology, we also want instances!
  \item Where do the instances come from? They come from software products and catalogues.
  \item How do we create \enquote{instances} in SNIK? From tables using Tarql.
  \item We don't have to reinvent the wheel, use reliable technology, save time and increase quality. 
  \item We did exactly that. Maryam supplied data and we transformed it to tables and then transformed the CSV tables to RDF using Tarql
  \item Each software product and each catalogue is stored as a separate file.
  \item $\Rightarrow$ \url{https://github.com/hitontology/csv2rdf} 
\end{itemize}
\end{frame}

\imageslide{}{Software Product I---Product Page}{img/productpage.png}
\imageslide{}{Software Product II---Extracted Attributes}{img/productpage.png}
\imageslide{}{Software Product III---Schematized Table}{img/productpage.png}
\imageslide{}{Software Product IV---Tarql}{img/productpage.png}
\imageslide{}{Software Product V---Repository}{img/productpage.png}
\imageslide{}{Software Product VI---SPARQL Upload}{img/productpage.png}
\imageslide{}{Software Product VII---Applications}{img/productpage.png}


\begin{frame}{HITO---CSV2RDF}
\begin{itemize}
  \item 
  \item 
  \item 
  \item 
\end{itemize}
\end{frame}

%\begin{frame}{Place in the Lifecycle}
%\centering
%\includegraphics[height=0.9\textheight]{cycle-red.pdf}
%\end{frame}

\begin{frame}{Questions}
\begin{itemize}
\item How is a \emph{new} software product best prepared to be included?
\item Why can I not find software product X?
\item I created a software product, why is it not in the endpoint?
\item Who is responsible for Y?
\item How do I edit an instance and what happens then?
\item At which places are software products stored?
\end{itemize}
\end{frame}

\begin{frame}{SPARQL Endpoint}
\centering
\begin{itemize}
\item \url{https://hitontology.eu/sparql}
\item like SNIK: powers all our applications
\item unlike SNIK: no permanent changes! everything you change there will be lost
\end{itemize}
\end{frame}

\begin{frame}{Ontology Repository}
\centering
\begin{itemize}
\item \url{https://github.com/hitontology/ontology}
\item this is where the data is stored permanently
\item we regularly overwrite the SPARQL endpoint with the repostory contents
\item combine script packs all turtle files into a single file that gets uploaded
\end{itemize}
\end{frame}

\begin{frame}{CSV2RDF: Tarql}
\centering
\begin{itemize}
\item
\end{itemize}
\end{frame}

\begin{frame}{}
\centering
\begin{itemize}
\item
\end{itemize}
\end{frame}

\begin{frame}{}
\centering
\begin{itemize}
\item
\end{itemize}
\end{frame}

\end{document}
