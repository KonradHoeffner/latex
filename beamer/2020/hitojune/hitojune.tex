\PassOptionsToPackage{rgb}{xcolor}
\documentclass[aspectratio=1610]{beamer}
\usepackage{tikz}
\usepackage[utf8]{inputenc}
\usepackage{booktabs}
\usepackage{tabulary}
\usepackage{amssymb}% http://ctan.org/pkg/amssymb
\usepackage{pifont}% http://ctan.org/pkg/pifont
\usepackage{csquotes}
\newcommand{\cmark}{\ding{51}}%
\newcommand{\xmark}{\ding{55}}%
\usetheme{simple}
\usecolortheme{whiteonblack}

\title{HITO Authoring Pipeline}
\subtitle{HITO Big Blue Button Meeting}
\author{Konrad Höffner \& Thomas Pause}

% Parameters: frame title, image path, (optional) note
\newcommand{\imageslide}[4][]
{
%\newgeometry{margin=0.0cm,top=1em}
\begin{frame}[plain]{~~~~#2}
\vspace{0.2em}
\centering\includegraphics[width=1.0\textwidth,height=0.95\textheight,keepaspectratio]{#3}
\\#1
\note{#4}
\end{frame}
%\restoregeometry
}

\begin{document}
\begin{frame}
\titlepage
\end{frame}

\begin{frame}{Lifting the Confusion}
\begin{tikzpicture}
%    \foreach \i in {1,...,30} {
\foreach [count=\count] \word in {Instance Generator, Ontology, Diagram, CSV2RDF, Tarql, Turtle, Catalogue, Classified, Citation, Issue, Protégé , ?, ?, ?, ?, ?, ?, ?, ?, ?} {
      \pgfmathparse{rnd}
      \definecolor{MyColor}{hsb}{\pgfmathresult,1,1}
      \pgfmathparse{3.0*rnd+1.0}
      \node[text=MyColor,
	rotate=rand*25]
	at (10*rnd,8*rnd) {\scalebox{\pgfmathresult}{\word}};
    };
  \end{tikzpicture}
\end{frame}

\begin{frame}{Questions}
\begin{itemize}
\item How is a \emph{new} software product best prepared to be included?
\item Why can I not find software product X?
\item I created a software product, why is it not in the endpoint?
\item Who is responsible for Y?
\item How do I edit an instance and what happens then?
\item At which places are software products stored?
\end{itemize}
\end{frame}

\begin{frame}{HITO---The Early Days}
\begin{itemize}
  \item at the beginning, all was well
  \item Verena modelled the HITO ontology in Protégé, and also some instance data
  \item HITO was living in a single file
  \item this single file was placed in a GitHub repository~\footnote{\url{https://github.com/hitontology/ontology}}, so that we have version control (backups, history and multi user editing)
  \item Verena: Protégé, Konrad: text editor (both on the GitHub repository), Franziska: yed diagram
\begin{itemize}
  \item Konrad and Verena are always synchronous through GitHub, but Franziska's diagram is \emph{not} synchronous! \emph{Problem:} some people think HITO "is" the diagram. But "the real HITO" is the file! (at this point). This problem can be solved by clarification.
  \item small diffs can be achieved by Konrad by editing in the Protégé way
\end{itemize}
  \item problems are solved, everything is great in ontology land!
  \item HITO is uploaded to a SPARQL endpoint, but it is not the master! Changes are not persistent there.
\end{itemize}
\end{frame}

\begin{frame}{HITO---Adolescence}
\begin{itemize}
  \item Problems are solved, everything is great in one file land?
  \item We don't only want ontology, we also want instances!
  \item Where do the instances come from? They come from software products and catalogues.
  \item How do we create \enquote{instances} in SNIK? From tables using Tarql.
  \item We don't have to reinvent the wheel, use reliable technology, save time and increase quality. 
  \item We did exactly that. Maryam supplied data and we transformed it to tables and then transformed the CSV tables to RDF using Tarql
  \item Each software product and each catalogue is stored as a separate file.
  \item $\Rightarrow$ \url{https://github.com/hitontology/csv2rdf} 
\end{itemize}
\end{frame}

\imageslide{Software Product I---Product Page}{img/bahmni-example2.png}{}
\imageslide{Software Product II---Supplied Table}{img/bahmni-xslx.png}{}
\imageslide{Software Product III---Polished Table}{img/bahmni-polished.png}{}
\imageslide{Software Product IV---Polished Table of Catalogue}{img/whodhi-single.png}{}
\imageslide{Software Product V---Polished Table of Catalogue}{img/whodhi-all.png}{}
\imageslide{Software Product VI---Tarql Mapping}{img/hitocsv2rdf.png}{}
\imageslide{Software Product VII---Repository}{img/hitogithub.png}{}
\imageslide{Software Product VIII---SPARQL Upload}{img/hitosparql.png}{}
\imageslide{Software Product IX---Applications}{img/lodview.png}{}

\begin{frame}{HITO---Adolescence Problems}
\begin{itemize}
  \item developer point of view: everything works perfectly
  \item successfully used for several catalogues and three test cases: Bahmni, Orthanc and GnuHealth
  \item domain expert point of view: large number of catalogues and entries, hard to find the correct ones for a certain citation
  \item domain experts: please tell us more problems from your perspective
\end{itemize}
\end{frame}

\begin{frame}{HITO---Instance Generator Phase}
\begin{itemize}
  \item was intended as a small tool to aid catalogue selection
  \item very limited scope on purpose---OntoWiki took 10 years to develop, for example \emph{no} direct export to SPARQL endpoint
  \item we are only 2 people and we don't have 10 years
  \item problem: increasing amount of requested features, large development effort
  \item problem: exported triples clash with existing workflow
  \item better fit with workflow would be to export table rows instead of triples and users need to put table rows in HITO medfloss tables
  \item still we think we have an intuitive interface to view and edit catalogue entries
  \item but different technologies to combine, needs to adhere to the right schema
  \item former consensus: Thomas gets the data as tables and uses whatever tools necessary to put instances into HITO
  \item then where is the data the domain experts provided?
\end{itemize}
\end{frame}

\begin{frame}{Where is the Roadblock---Data Quality}
\begin{itemize}
  \item domain experts use different approaches and suboptimal solutions to prepare data, e.g. different ways to express citation to catalogue entry mappings
  \item solution: domain experts should use the predefined schema of the HITO tables
  \item then Thomas can create new tables using the existing scheme and use CSV2RDF with Konrad to add new instances to the ontology
  \item note: Many of the tables given to Thomas don't contain classified entries mapped to the citations. Thomas has no way to enter citations with missing classified catalogue entries (except unknown). So Thomas can only start when he is given a complete mapping between citations and classified entries.
  \item then the domain experts should evaluate the finished instances using the yet to be developed integrated viewer and propose changes
\end{itemize}
\end{frame}

\begin{frame}{Discussion}
\end{frame}

\end{document}
