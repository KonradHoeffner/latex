%\documentclass[aspectratio=43]{beamer}
\documentclass[aspectratio=169]{beamer}
%\documentclass[aspectratio=1610,12pt]{beamer}
\usepackage[utf8]{inputenc}
\usepackage[T1]{fontenc}
\usepackage{lmodern}

%\usepackage[english]{babel}
\usepackage[ngerman]{babel} %use this for German presentations
\usepackage{booktabs} % fancy tables
\usepackage{csquotes}
\usepackage{tabulary} % tables with auto column length
\usepackage{hyperref}

\usetheme{imise2}
\author{Konrad Höffner}
\date{Leipzig, 12. Juni 2020}
\title{Wissenschaftsseminar}
\subtitle{}
\def\address{Härtelstraße 16-18, 04107 Leipzig, Raum 227}
\def\email{konrad.hoeffner@imise.uni-leipzig.de}
\def\telephone{+49 341 97 16322}

\begin{document}
\begin{frame}
\titlepage
\end{frame}

\begin{frame}{Agenda}
\begin{enumerate}
\item Hinweise 
\item Zeitplan
\item Thomas Pause Zweiter Bachelorvortrag\\\enquote{Organisation von Teilgraphen semantischer Netze zur ontologiegestützten Wissensvermittlung}
\item Martin Schöbel Erster Mastervortrag\\\enquote{Konzeption und Umsetzung des Erweiterungsmoduls Strukturdaten des Kerndatensatzes der Medizininformatikinitiative}
\end{enumerate}
\end{frame}

\begin{frame}{Hinweise}
\begin{itemize}
\item Abschlussarbeiten-Kurse zusammengelegt in \\ \enquote{Abschlussarbeiten Medizinische Informatik} (\url{https://moodle2.uni-leipzig.de/course/view.php?id=18063})
\item Nachrichten Mailingliste $\rightarrow$ Moodle
\item Während Corona Big Blue Button\\\url{https://moodle2.uni-leipzig.de/mod/bigbluebuttonbn/view.php?id=1081692}
\item Wissenschaftsseminar einmal im Monat, $4 \rightarrow 3$ Zeitfenster
\item 15 Minuten $\rightarrow$ 20 Minuten 
%\item 
\item Trainer: Achtung bei Doppelrolle
\end{itemize}
\end{frame}

\begin{frame}{Zeitplan}
Bitte erst X Wochen nach Abgabe und Akzeptanz der Einleitung vortragen:
\begin{tabulary}{\textwidth}{llL}
\toprule
                   & Woche & Inhalt                                                                                           \\
\midrule
Einleitungsvortrag & 4                              & Aufgabenstellung, Stand der Forschung (Literaturrecherche), Projektplan                          \\
Zwischenvortrag    & 8                              & Bei praktischer Arbeit: Einzusetzende Methodik. Bei theoretischer Arbeit: Grundlegende Methoden. \\
Abschlussvortrag   & 21/24     & Ziele und Ergebnisse, Diskussion, Ausblick\\
\bottomrule
\end{tabulary}
\end{frame}

%\begin{frame}{Fragen}
%\end{frame}

\end{document}
