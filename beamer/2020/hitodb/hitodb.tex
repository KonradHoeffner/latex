%\documentclass[aspectratio=43]{beamer}
\documentclass[aspectratio=169]{beamer}
%\documentclass[aspectratio=1610,12pt]{beamer}
\usepackage[utf8]{inputenc}
\usepackage[T1]{fontenc}
\usepackage{lmodern}

%\usepackage[english]{babel}
\usepackage[ngerman]{babel} %use this for German presentations
\usepackage{booktabs} % fancy tables
\usepackage{csquotes}
\usepackage{tabulary} % tables with auto column length
\usepackage{hyperref}
\usepackage{amssymb}

\usetheme{imise2}
\author{Konrad Höffner}
\date{Leipzig, 12. Juni 2020}
\title{HITO Datenbankprojekt}
\subtitle{}
\def\address{Härtelstraße 16-18, 04107 Leipzig, Raum 227}
\def\email{konrad.hoeffner@imise.uni-leipzig.de}
\def\telephone{+49 341 97 16322}

\begin{document}
\begin{frame}
\titlepage
\end{frame}

\begin{frame}{Gegenstand}
\begin{itemize}
\item Ein Teil von HITO befasst sich mit Softwareprodukten
\item Domänenexperten müssen Softwareprodukte anzeigen, hinzufügen und editieren.
\item Dafür soll eine Datenbank genutzt werden.
\end{itemize}
\end{frame}

\begin{frame}{Problemstellung}
\begin{itemize}
\item Derzeit werden Semantic Web Tools eingesetzt.
\item Ohne die Kataloge könnte man einfach OntoWiki nehmen aber die Citation-Classified-Catalogue Struktur wird nicht gut abgebildet.
\end{itemize}
\end{frame}

\begin{frame}{Motivation}
\begin{itemize}
\item Oracle Forms und co. versprechen intuitive Bearbeitung von Softwareprodukten 
\item Datenbanken gibt es schon länger als das Semantic Web, wir nehmen an, dass es dort besser ausgereifte Werkzeuge gibt
\item Resultat ist sofort vorhanden bzw. aktualisiert, wesentlich intuitiverer Arbeitsablauf als mit Instancegenerator \& GitHub Issues.
\end{itemize}
\end{frame}
\begin{frame}{Ziele}
\begin{itemize}
\item Z1 Datenbank aufsetzen \checkmark
\item Z2 RDF zu Datenbank konvertieren
\item Z3 Werkzeuge bereitstellen
\item Z4 DB zurück zu RDF konvertieren
\end{itemize}
\end{frame}

\begin{frame}{Aufgaben Z1 Datenbank aufsetzen}
\begin{itemize}
\item A1.1 Datenbanksystemanalyse, -bewertung und -auswahl -> PostgreSQL \checkmark 
\item A1.2 Datenbanksystemeinführung\\-> Installation als Docker container auf Bruchtal \checkmark  
\item A1.3 Schema erstellen (July 1--July 8) \checkmark
\end{itemize}
\end{frame}

\begin{frame}{Aufgaben Z2 RDF zu Datenbank konvertieren}
\begin{itemize}
\item A2.1 externe Instanzlisten laden (DBpedia Language, Programming Language, SWO License) \checkmark
\item A2.2 interne Instanzlisten laden (Interoperability Standard) \checkmark
\item A2.3 Katalogtabellen von Google Sheets als CSV laden \checkmark
\item A2.4 Softwareprodukte von Google Sheets laden
\end{itemize}
\end{frame}

\begin{frame}{Aufgaben Z3 Werkzeuge bereitstellen}
\begin{itemize}
\item A3.1 Analyse und Bewertung existierender Form-Generatoren
\item A3.2 Einführung Form-Generator
\item A3.3 ?
\end{itemize}
\end{frame}

\begin{frame}{Aufgaben Z4 DB zurück zu RDF konvertieren}
\begin{itemize}
\item Systemanalyse und -bewertung (Tarql?)
\item Einführung
\end{itemize}
\end{frame}

\begin{frame}{Arbeitspakete und Meilensteine A3.1 Form-Generatoren}
\begin{itemize}
\item TODO Thomas add here please
\item 
\end{itemize}
\end{frame}

\iffalse
\begin{frame}{Arbeitspakete und Meilensteine}
\begin{itemize}
\item 
\item 
\end{itemize}
\end{frame}
\fi

\end{document}
