\documentclass[14pt,aspectratio=1610]{beamer}
\usepackage[T1]{fontenc}
\usepackage[utf8]{inputenc}
\usepackage[ngerman]{babel}
\uselanguage{ngerman}
\languagepath{ngerman}
\usetheme{simple}
\usecolortheme{whiteonblack}
\usepackage{listings}
\usepackage{booktabs}
\usepackage{csquotes}
\usepackage{url}
\usepackage{tabulary}
%\usepackage{graphicx}
%\date{25. März 2022}
\title{Einfacher Ontologieexperte für Domänenexperten}
\subtitle{Analyse}
\begin{document}

\begin{frame}
\titlepage
\end{frame}

\begin{frame}{Gegenstand}
\begin{itemize}
\item kleinere Forschungsprojekte mit Ontologie als Forschungsergebnis
\item Beispiel ANNO: Antopological Notation Ontology
\end{itemize}
\end{frame}

\begin{frame}{Ziele}
\begin{itemize}
\item Qualität möglichst hoch
\item Aufwand möglichst niedrig
\end{itemize}
\end{frame}

\begin{frame}{Probleme}
\begin{itemize}
\item Meist nicht alle Fähigkeiten in einer Person vereint.
\item Gruppen: Domänenexperten und Ontologen
%\item Untergruppen: Top Level Ontologen, Semantic Web Techniker, Entwickler
\item Domänenexperten wissen nicht, wie Ontologien funktionieren.
\item Ontologen kennen die Domäne nicht.
\item Manchmal sogar in unterschiedlichen Unis und Städten.
\end{itemize}
%\begin{tabular}{lccc}
%Gruppe				&Domänenexperten	&Ontologen	&Entwickler\\
%Domäne				&++\\
%Ontologien		&
%Entwickler		&
%\end{tabular}
\end{frame}

\begin{frame}[fragile]{Persona I}
\begin{tabulary}{\textwidth}{lL}
%Name, Alter		&Lisa Müller, 27\\
Hintergrund		&Paläontologie\\
Beruf					&Doktorandin\\
Situation			&Extrem gestresst, ist in Forschungsprojekt zu Ontologie muss aber nebenbei noch Lehre, Doktorarbeit und viele andere Aufgaben übernehmen.\\
Ziel					&Möchte möglichst schnell die Ontologie fertig bekommen, Qualität und Vollständigkeit nicht so wichtig.\\
\end{tabulary}
\end{frame}

\begin{frame}[fragile]{Persona II}
\begin{tabulary}{\textwidth}{lL}
%Name, Alter		&Boris , 27\\
Hintergrund		&Ontologien\\
Beruf					&Gruppenleiter\\
Situation			&Möchte möglichst viele erfolgreiche Projekte beantragen und durchführen.\\
Ziel					&Hohe Qualität der Ontologie, Anbindung an TLO, Reasoning, Axiome, Publikation in Fachjournal. Ontologie im Vordergrund.\\
\end{tabulary}
\end{frame}

\begin{frame}[fragile]{Persona III}
\begin{tabulary}{\textwidth}{lL}
Hintergrund		&Semantic Web\\
Beruf					&Wissenschaftlicher Mitarbeiter\\
Situation			&Arbeitet in immer neuen Ontologieprojekten mit und erkennt Gemeinsamkeiten und Probleme.\\
Ziel					&Veröffentlichung nach Semantic Web Standards, elegante Modellierung mit möglichst wenig Tripeln pro Aussage, Performance, Interlinks. Ontologie als leichtgewichtiges Schema.\\
\end{tabulary}
\end{frame}

\begin{frame}[fragile]{Persona IV}
\begin{tabulary}{\textwidth}{lL}
Hintergrund		&Medizininformatik\\
Beruf					&Wissenschaftlicher Mitarbeiter\\
Situation			&Softwareentwicklung für Forschungsprojekte, Publikationen, $\ldots$\\
Ziel					&möglichst alles in Docker, wiederverwendbare Komponenten, APIs, würde Ontologie am liebsten einfach im Texteditor schreiben.\\
\end{tabulary}
\end{frame}

\begin{frame}[fragile]{Ansätze}
\small
\begin{tabular}{lcccc}
											&Domänenexp.	&Ontologen	&\enquote{Semantic Webber}	&Entwickler\\
Protégé								&--								&++				&$\cdot{}$									&$\cdot$\\
CRUD									&+++							&+				&+													&---\\
LLM										&+++							&?				&?													&noch ---\\
SMOG									&+								&+				&+													&+\\
\midrule
SMOG GUI							&++								&+				&+													&$\cdot$\\
\bottomrule
\end{tabular}
\end{frame}

\begin{frame}{Anderes}
\begin{itemize}
\item \url{https://www.w3.org/wiki/Ontology\_editors}
\item \url{https://github.com/semantalytics/awesome-semantic-web}
\item dosdp-tools: CLI
\item RDFSharp/OWLSharp: Framework für .NET applications
\item Grafo: visueller designer, kostenpflichtige und freie Version, sign up ging nicht
\item Cameo Concept Modeler: kommerziell
\item SWOOP: Link ist tot
\item Neologism: Fokus auf Vokabulare, Demo ging nicht
\end{itemize}
\end{frame}

\begin{frame}{SMOG GUI Ziele}
\begin{itemize}
\item einfach zu benutzen für Domänenexperten ohne Ontologie- und Semantic Web Expertise
\item Properties definieren
\item Hierarchieven von und Interlinks zwischen Klassen
\item MUSS in Kooperation benutzt werden
\item SOLLTE ohne Installation funktionieren
\item kleines Projekt, vertretbarer Entwicklungsaufwand
\end{itemize}
\end{frame}

\begin{frame}{Warum nicht einfach (Web) Protégé}
\begin{itemize}
\item mehr Funktionalität
\item schneller und einfacher für nicht-Ontologie-Experten benutzbar
\item Ontologiemanagement: Organisationen, Repositories, $\ldots$
\end{itemize}
\end{frame}

\begin{frame}{Konzept}
\begin{itemize}
\item Entscheidung gegen modulare Webanwendung: zu viel Aufwand
\item Da SMOG in Java geschrieben ist auch SMOG GUI
\item JavaFX Anwendung
\item Editierung Konfiguration und das was jetzt in Tabelle ist (Hierarchie und Inhalte)
\end{itemize}
\end{frame}

\begin{frame}{Stand jetzt}
\begin{itemize}
\item \url{https://github.com/KonradHoeffner/SMOG/tree/gui}
\item Konfiguration laden
\end{itemize}
\end{frame}

\begin{frame}{Zusatz}
\begin{itemize}
\item \url{http://owlgred.lumii.lv/online\_visualization/pizza2.owl}
\item Web Protege erweitern?
\item muss nicht auf SMOG aufbauen
\item im Juli weiter besprechen
\end{itemize}
\end{frame}

\end{document}
