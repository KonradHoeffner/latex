\documentclass[14pt,aspectratio=1610]{beamer}
\usepackage[T1]{fontenc}
\usepackage[utf8]{inputenc}
\usepackage[ngerman]{babel}
\uselanguage{ngerman}
\languagepath{ngerman}
\usetheme{simple}
\usecolortheme{whiteonblack}
\usepackage{listings}
\usepackage{csquotes}
\usepackage{url}
%\usepackage{graphicx}
%\date{25. März 2022}
\title{Markdown \& GitHub Pages Workshop}
\subtitle{Schnell und einfach statische Webseiten erstellen}
\begin{document}

\begin{frame}
\titlepage
\end{frame}

\begin{frame}{Voraussetzungen}
\begin{itemize}
\item PC, am besten zwei Monitore
\item GitHub Account
\item simple Inhalte: Texte, Bilder, Tabellen
\item Markdown Cheat Sheet, z.B. \url{https://oddmuse.org/wiki/Markdown\_Cheat\_Sheet}
\item optional: Git, Discount, Ruby, Bundler
\end{itemize}
\end{frame}

\begin{frame}{Motivation}
\begin{itemize}
\item Krankenhaus:\\\enquote{Was nicht dokumentiert wird, ist nicht passiert.}
\item Internet:\\\enquote{Was keine Webseite hat, existiert nicht.}
\item Nicht jedes kleine Projekt braucht Social Media aber wenigstens eine kleine Seite.
\item Dokumentationen, Linksammlungen.
%\item Beispiel: \url{https://github.com/semantalytics/awesome-semantic-web}
\end{itemize}
\end{frame}

\begin{frame}{CMS?}
\begin{itemize}
\item WordPress:\\umständlich, langsam, \enquote{Free}-Paket mit Werbung
\item selbst hosten: Datenbank, PHP, $\ldots$ $\rightarrow$ zusätzlicher Aufwand und Ressourcen
\item Overkill für simple, statische Inhalte
%\item Webspace und Domain mieten, Seite aktuell halten:\\Geld und Arbeit
\end{itemize}
\end{frame}

\begin{frame}{HTML \& CSS selbst schreiben?}
\scriptsize
\lstinputlisting[language=html]{test.html}
\end{frame}

\begin{frame}[fragile]{Markdown}
\small
\lstinputlisting{test.md}
\end{frame}

\begin{frame}{Lernziele}
\begin{enumerate}
\item GitHub repository erstellen
\item Dokument in Markdown schreiben
\item Dokument als Webseite veröffentlichen
\end{enumerate}
\end{frame}

\begin{frame}{1. GitHub repository erstellen}
\begin{itemize}
\item oben rechts \enquote{+} $\rightarrow$ new repository
\item \textbf{public}
\item \enquote{Add a README file}
\end{itemize}
\end{frame}

\begin{frame}[fragile]{2. Dokument in Markdown schreiben}
\begin{itemize}
\item README.md $\rightarrow$ edit this file
\item Aufgabe: kurzen Text mit Formatierungen schreiben
\end{itemize}

~\\
\emph{Optional in \texttt{.bashrc} / \texttt{.zshrc}:}\\
\begin{verbatim}
function mdview() {
markdown -f fencedcode "\@" > /tmp/markdown.html;
firefox /tmp/markdown.html }
\end{verbatim}

\url{https://github.com/KonradHoeffner/dotfiles/blob/master/.zshrc}
\end{frame}

\begin{frame}[fragile]{Erinnerung}
\small
\lstinputlisting{test.md}
\end{frame}

\begin{frame}{3. Webseite veröffentlichen}
\begin{itemize}
\item Settings $\rightarrow$ Code and Automation $\rightarrow$ Pages $\rightarrow$ Branch $\rightarrow$ master/main $\rightarrow$ Save
\item kurz warten
\item Seite neuladen
\item \texttt{https://username.github.io/reponame/}
\item Tipp: Code $\rightarrow$ About $\rightarrow$ Use your GitHub Pages website
\end{itemize}
\end{frame}

\begin{frame}{Use Case BaseTRACE}
\begin{itemize}
\item \url{https://github.com/medizininformatik-initiative/BaseTRACE}
\item Fork erstellen
\item Settings $\rightarrow$ Code and Automation $\rightarrow$ Pages $\rightarrow$ Branch $\rightarrow$ master/main $\rightarrow$ Save
\item z.B. \texttt{https://konradhoeffner.github.io/BaseTRACE/}
\end{itemize}
\end{frame}

\begin{frame}{Hintergrund}
\begin{itemize}
\item GitHub Workflow wird ausgelöst
\item Jekyll: Markdown $\rightarrow$ static HTML
\end{itemize}
\end{frame}

\begin{frame}{Themes}
\begin{itemize}
\item Beispiel: Leap Day \url{https://github.com/pages-themes/leap-day}
\item ...
\end{itemize}
\end{frame}

\iffalse
\begin{frame}{Varianten}
\begin{itemize}
\item GitHub Flavored Markdown (GFM) $\supset$ CommonMark
\item Jekyll: Kramdown 
\end{itemize}
\end{frame}

\begin{frame}{Use Case BaseTRACE}
\begin{itemize}
\item \url{https://github.com/medizininformatik-initiative/BaseTRACE}
\item Fork erstellen
\item Settings $\rightarrow$ Code and Automation $\rightarrow$ Pages $\rightarrow$ Branch $\rightarrow$ master/main $\rightarrow$ Save
\item z.B. \texttt{https://konradhoeffner.github.io/BaseTRACE/}
\end{itemize}
\end{frame}

\begin{frame}{Links}
\begin{itemize}
\item \url{https://www.w3.org/TR/shacl/}
\item \url{https://www.ldf.fi/service/owl-rl-reasoner}
\item \url{https://github.com/konradhoeffner/shacl}
\end{itemize}
\end{frame}
\fi

\end{document}
