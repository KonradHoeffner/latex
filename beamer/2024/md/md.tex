\documentclass[14pt,aspectratio=1610]{beamer}
\usepackage[T1]{fontenc}
\usepackage[utf8]{inputenc}
\usepackage[ngerman]{babel}
\uselanguage{ngerman}
\languagepath{ngerman}
\usetheme{simple}
\usecolortheme{whiteonblack}
\usepackage{listings}
\usepackage{csquotes}
\usepackage{url}
%\usepackage{graphicx}
%\date{25. März 2022}
\title{Markdown \& GitHub Pages Workshop}
\subtitle{Schnell und einfach statische Webseiten erstellen}
\begin{document}

\begin{frame}
\titlepage
\end{frame}

\begin{frame}{Voraussetzungen}
\begin{itemize}
\item PC
\item GitHub Account
\item simple Inhalte: Texte, Bilder, Tabellen
\end{itemize}
\end{frame}

\begin{frame}{Motivation}
\begin{itemize}
\item Krankenhaus:\\\enquote{Was nicht dokumentiert wird, ist nicht passiert.}
\item Internet:\\\enquote{Was keine Webseite hat, existiert nicht.}
\item Nicht jedes kleine Projekt braucht Social Media aber wenigstens eine kleine Seite.
\item Dokumentationen, Linksammlungen.
\item Beispiel: \url{https://github.com/semantalytics/awesome-semantic-web}
\end{itemize}
\end{frame}

\begin{frame}{HTML \& CSS selbst schreiben?}
\scriptsize
\lstinputlisting[language=html]{test.html}
\end{frame}

\begin{frame}{CMS?}
\begin{itemize}
\item WordPress:\\umständlich, langsam, \enquote{Free}-Paket mit Werbung
\item selbst hosten: Datenbank, PHP, $\ldots$ $\rightarrow$ zusätzlicher Aufwand und Ressourcen
\item Overkill für simple, statische Inhalte
%\item Webspace und Domain mieten, Seite aktuell halten:\\Geld und Arbeit
\end{itemize}
\end{frame}

\begin{frame}[fragile]{Markdown}
\small
\lstinputlisting{test.md}
\end{frame}

\begin{frame}{Lernziele}
\begin{enumerate}
\item GitHub repository erstellen
\item Dokument in Markdown schreiben
\item Dokument als Webseite veröffentlichen
\item Webseite gemeinsam bearbeiten
\end{enumerate}
\end{frame}

\begin{frame}{1. GitHub repository erstellen}
\begin{itemize}
\item oben rechts \enquote{+} $\rightarrow$ new repository
\item public
\item \enquote{Add a README file}
\end{itemize}
\end{frame}

\begin{frame}{2. Dokument in Markdown schreiben}
\begin{itemize}
\item README.md $\rightarrow$ edit this file
\item Aufgabe: kurzen Text mit Formatierungen schreiben
\end{itemize}
\end{frame}

\begin{frame}[fragile]{Erinnerung}
\small
\lstinputlisting{test.md}
\end{frame}

\begin{frame}{3. Webseite veröffentlichen}
\begin{itemize}
\item Settings $\rightarrow$ Code and Automation $\rightarrow$ Pages $\rightarrow$ Branch $\rightarrow$ master $\rightarrow$ Save
\item 
\item 
\end{itemize}
\end{frame}

\begin{frame}{1. GitHub repository erstellen}
\begin{itemize}
\item oben rechts \enquote{+} $\rightarrow$ new repository
\item public
\item \enquote{Add a README file}
\end{itemize}
\end{frame}



\begin{frame}{Links}
\begin{itemize}
\item \url{https://www.w3.org/TR/shacl/}
\item \url{https://www.ldf.fi/service/owl-rl-reasoner}
\item \url{https://github.com/konradhoeffner/shacl}
\end{itemize}
\end{frame}

\end{document}
