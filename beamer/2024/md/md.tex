\documentclass[14pt,aspectratio=1610]{beamer}
\usepackage[T1]{fontenc}
\usepackage[utf8]{inputenc}
\usepackage[ngerman]{babel}
\uselanguage{ngerman}
\languagepath{ngerman}
\usetheme{simple}
\usecolortheme{whiteonblack}
\usepackage{listings}
\usepackage{csquotes}
\usepackage{url}
%\usepackage{graphicx}
%\date{25. März 2022}
\title{Markdown \& GitHub Pages Workshop}
\subtitle{Schnell und einfach statische Webseiten erstellen}
\begin{document}

\begin{frame}
\titlepage
\end{frame}

\begin{frame}{Vorraussetzungen}
\begin{itemize}
\item PC
\item GitHub Account
\item simple Inhalte: Texte, Bilder, Tabellen
\end{itemize}
\end{frame}

\begin{frame}{Motivation}
\begin{itemize}
\item Krankenhaus:\\\enquote{Was nicht dokumentiert wird, ist nicht passiert.}
\item Internet:\\\enquote{Was keine Webseite hat, existiert nicht.}
\item nicht jedes kleine Projekt braucht Social Media aber wenigstens eine kleine Seite
\end{itemize}
\end{frame}

\begin{frame}{Alternativen}
\begin{itemize}
\item HTML, CSS, JavaScript \& co selbst schreiben: umständlich
\item WordPress:\\umständlich, langsam, \enquote{Free}-Paket mit Werbung
\item Webspace und Domain mieten, Seite aktuell halten:\\Geld und Arbeit
\end{itemize}
\end{frame}

\begin{frame}{Lernziele}
\begin{enumerate}
\item GitHub repository erstellen
\item Dokument in Markdown schreiben
\item Dokument als Webseite veröffentlichen
\item Webseite gemeinsam bearbeiten
\end{enumerate}
\end{frame}

\begin{frame}[fragile]{Markdown}
\small
\lstinputlisting{test.md}
\end{frame}


\begin{frame}{Links}
\begin{itemize}
\item \url{https://www.w3.org/TR/shacl/}
\item \url{https://www.ldf.fi/service/owl-rl-reasoner}
\item \url{https://github.com/konradhoeffner/shacl}
\end{itemize}
\end{frame}

\end{document}
