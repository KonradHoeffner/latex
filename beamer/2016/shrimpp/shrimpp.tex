\documentclass[14pt]{beamer}
\usepackage[utf8]{inputenc} 
\usetheme{simple}
\usecolortheme{whiteonblack}
%\usepackage{array}
%\usepackage{hyperref}
%\usepackage{csquotes}

%\author{Konrad Höffner}
%\title{Resource Description Framework (RDF) und Ontologien}
%\subtitle{Kurze Einführung mit Vorteilen für Shrimpp}

\begin{document}

%\begin{frame}
%\titlepage
%\end{frame}

\section{Was ist RDF?}

\begin{frame}{Tripel}
\begin{itemize}
\item RDF Datenmodell basiert dem Tripel (3-Tupel)
\item Ein Tripel repräsentiert eine simple Aussage
\item Bestandteile: Subjekt, Prädikat, Objekt
\end{itemize}
\end{frame}

\begin{frame}{Subjekt}
\begin{itemize}
\item Subjekt wird mittels URI (Uniform Resource Identifier) identifiziert
\item Beispiel: \url{http://dbpedia.org/resource/Leipzig}
\item Ein Subjekt bezeichnet ein beliebiges Ding (eine \emph{Ressource})
\end{itemize}
\end{frame}

\begin{frame}{Objekt}
\begin{itemize}
\item Entweder eine Ressource (ein anderes Ding) oder ein Literal (eine Zeichenkette) 
\item \url{http://dbpedia.org/resource/Burkhard\_Jung}
\item \enquote{POINT(12.383333206177 51.333332061768)}
\end{itemize}
\end{frame}

\begin{frame}{Prädikat}
\begin{itemize}
\item auch eine Ressource (mittels URI identifiziert)
\item eine zweistellige Relation, die Subjekt und Objekt verbindet
\item \url{http://dbpedia.org/property/mayor}
\item \url{http://www.w3.org/2003/01/geo/wgs84\_pos#geometry} 
\end{itemize}
\end{frame}

\begin{frame}{Beispieltripel}
\begin{itemize}
\item \url{http://dbpedia.org/resource/Leipzig} \url{http://dbpedia.org/property/mayor} \url{http://dbpedia.org/resource/Burkhard\_Jung}.
\item y
\end{itemize}
\end{frame}

\begin{frame}{Symmetrie}
\begin{itemize}
\item x
\item y
\end{itemize}
\end{frame}

\begin{frame}{Was }
\only<1>
{
\begin{tabular}{ll}
\end{tabular}
}
\only<2>
{
\begin{tabular}{ll}
\end{tabular}
}
\end{frame}

\begin{frame}{}
\begin{block}{}
\begin{itemize}
\item x
\item y
\end{itemize}
\end{block}
\end{frame}

\begin{frame}{}
\begin{itemize}
\item x
\item y
\end{itemize}
\end{frame}

\section{Reasoning}

\begin{frame}{Was ist Reasoning?}
\begin{itemize}
\item Inferieren von Impliziter Information aus vorhandener Fakten und Relationseigenschaften
\item Beispiel: Alice ist Mutter von Trudy. Trudy ist Mutter von Bob. $\rightarrow$ Alice ist Großmutter von Bob. 
\end{itemize}
\end{frame}

\begin{frame}{Symmetrie}
\begin{itemize}
\item x
\item y
\end{itemize}
\end{frame}

\begin{frame}{}
\begin{itemize}
\item x
\item y
\end{itemize}
\end{frame}

\end{document}
