%\documentclass[aspectratio=43]{beamer}
%\documentclass[aspectratio=169]{beamer}
\documentclass[aspectratio=1610,12pt]{beamer}

\usepackage[english]{babel}
%\usepackage[ngerman]{babel} %use this for German presentations
\usepackage{booktabs} % fancy tables
\usepackage{tabulary} % tables with auto column length 
\usepackage{hyperref}

\newcommand{\imageslide}[4][]
{
\newgeometry{margin=0cm,top=1em}
\begin{frame}[plain]{~~~~#2}
\vspace{0.2em}
\centering\includegraphics[width=1.0\textwidth,height=0.95\textheight,keepaspectratio]{#3}
\\#1
\note{#4}
\end{frame}
\restoregeometry
}

\usetheme{imise2}
\author{Konrad Höffner}
\date{Leipzig, June 26, 2019}
\title{The SNIK Web Quiz}
\subtitle{\url{https://www.snik.eu/quiz}}
\def\address{Härtelstraße 16-18, 04107 Leipzig, Raum 227}
\def\email{konrad.hoeffner@imise.uni-leipzig.de} 
\def\telephone{+49 341 97 16322}
%\url{https://github.com/IMISE/imise-beamertheme/issues}
\begin{document}
\begin{frame}
\titlepage
\end{frame}

\imageslide{Example}{img/snik-quiz-screenshot.png}{}{}

\begin{frame}
\frametitle{How to Play SNIK Quiz}
\begin{enumerate}
\item Open your browser:\\ Chrome and Firefox work, Internet Explorer doesn't, others are untested\footnote{Please let us know at \url{https://github.com/imise/snik-quiz/issues}}.
\item Go to \url{https://www.snik.eu/quiz}
\item Answer the questions in time. 
\item (optional) Submit your score. The highest score wins a prize! 
\end{enumerate}
\end{frame}

\begin{frame}{Scoring}
\begin{itemize}
\item Choosing the right answer on first try gives 23 points, 2nd try 11 points, 3rd 5 points and 4th no point. 
\item Don't rush, the time does not influence the score as long as you don't use all of it.
\item Be careful about subject and object!\\"What contains an Order Entry" $\neq$ "What does an Order Entry contain"!
\end{itemize}
\end{frame}

\begin{frame}{Known Issues}
\begin{itemize}
\item Prototype, may contain bugs.
\item The questions are automatically created using the ontology and may not sound intuitive.
\item There may be identically looking answers to the same questions from different ontologies, you can only guess there.
\end{itemize}
\end{frame}

\end{document}
