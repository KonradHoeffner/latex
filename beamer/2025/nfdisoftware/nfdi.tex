% 2025-03-07 IMISE Colloqium
% 15 minutes, Deutsch, 16:10
% Funded by DFG as part of NFDI. Grant Number:  521466146
%\documentclass[aspectratio=43]{beamer}
%\documentclass[aspectratio=169]{beamer}
\documentclass[aspectratio=1610,12pt]{beamer}

\usepackage[ngerman]{babel}
\usepackage{booktabs} % fancy tables
\usepackage{tabulary} % tables with auto column length 
\usepackage{hyperref}
\usepackage{siunitx}

\usetheme{imise}
\author{\emph{Matthias Löbe}, Konrad Höffner}
%\title{The SNIK Ontology of Information Management in Hospitals}
\title{\large nfdi.software – ein zentraler Marktplatz für Forschungssoftware}
\date{7. März 2025, 14:55--15:10 Uhr}
\def\address{Härtelstraße 16-18, 04107 Leipzig, Raum 227}
%\def\email{konrad.hoeffner@imise.uni-leipzig.de} 
\def\email{matthias.loebe@imise.uni-leipzig.de} 
%nfdi-software@lists.nfdi.de
%\def\telephone{+49 341 97 16322}
\def\telephone{+49 341 97 16 113}


\newcommand{\imageslide}[4][]
{
\newgeometry{margin=0cm,top=1em}
\begin{frame}[plain]{~~~~#2}
\vspace{0.2em}
\centering\includegraphics[width=0.995\textwidth,height=0.95\textheight,keepaspectratio]{#3}
\\#1
\note{#4}
\end{frame}
\restoregeometry
}

\newcommand{\imageslidebare}[1]
{
\newgeometry{margin=0cm,top=1em}
\begin{frame}[plain]
\centering\includegraphics[width=0.995\textwidth,height=0.995\textheight,keepaspectratio]{#1}
\end{frame}
\restoregeometry
}

\begin{document}

\begin{frame}
\titlepage
\end{frame}

\begin{frame}{Forschungssoftware}
\begin{itemize}
\item Definition: Für Forschung verwendet oder im Rahmen von Forschung entstanden
\item Bedarf an nachhaltiger Nutzung und Weiterentwicklung wächst in Kultur-, Geistes-, Sozial-, Ingenieur-, Natur- und Lebenswissenschaften
\item Forschungssoftware finden
\item Forschungssoftware evaluieren und für eigene Forschung verwenden
\end{itemize}
\end{frame}

\begin{frame}{Forschungssoftwaremetadaten}
\begin{itemize}
\item semantisch strukturierte Metadaten für Forschungssoftware
\item Citation File Format (CFF)
\item CodeMeta
\item Verbindungen zu Forschungsdaten herstellen
\end{itemize}
\end{frame}

\begin{frame}{nfdi.software}
\centering\includegraphics[width=0.5\textwidth,height=0.95\textheight,keepaspectratio]{img/logo.png}
\begin{itemize}
\item Nationale Forschungsdateninfrastruktur (NFDI): große Menge an Konsortien
\item zentraler Einstiegspunkt für Forschungssoftware im NFDI bzw. Deutschland
\item Forschungssoftware FAIRifizieren
\item Konsortien und wissenschaftliche Communities integrieren
\item Metadaten extrahieren, standardisieren und anreichern
\item Mehrfachentwicklungen verhindern, Wiederverwendung fördern
\item Kompatibilität erhöhen
\end{itemize}
\end{frame}

\begin{frame}{Ansatz}
\begin{itemize}
\item existierende Quellen und Pläne erfassen, verbinden, verbessern und wiedergeben:
\item Infrastruktur entwerfen, implementieren und testen
\item Benutzerfeedback ermöglichen
\item schrittweise Entwicklung eines Prototyps
\end{itemize}
\end{frame}

\begin{frame}{Quellen}
\begin{itemize}
\item Software-Register
\item Marktplätze
\item Metadaten-Repositorien (z.B. physics.tools) und Code-Repositorien (z.B. GitHub)
\item Publikationen
\end{itemize}
\end{frame}

\begin{frame}{Existierende und geplante Komponenten}
\begin{itemize}
\item Research Software Directory (RSD): fachgebietsübergreifend Forschungssoftware finden, zitieren und wiederverwenden
\item ELIXIR Research Software Ecosystem (RSEc \& bio.tools): Software für Bioinformatik finden und vergleichen
\item Betty’s Research Engine: Suchmaschine für Code-Repositories und damit verbundene Publikationen
\item Physics.tools: Softwaresuchmaschine für arxiv-Publikationen, ermittelt Metadaten aus dem Code-Repository
\item LLM-basierte Suchmaschine: Software für einen Zweck intuitiv ermitteln, ähnliche Produkte finden
\end{itemize}
\end{frame}

\iffalse
\begin{frame}[fragile]{Questions?}
\begin{itemize}
%\item Diese Präsentation \url{https://github.com/KonradHoeffner/latex/releases/download/colloquium/colloquium.pdf}
\vspace{0.5em}%here it works as intended
\item Projekt \url{https://base4nfdi.de/projects/nfdi-software}
\item Proposal \url{https://doi.org/10.5281/zenodo.14507301}
\item Contact \url{konrad.hoeffner@imise.uni-leipzig.de}
\end{itemize}
\end{frame}
\fi

\end{document}
