\documentclass[aspectratio=1610]{beamer}
\usepackage[utf8]{inputenc}
\usepackage{booktabs}
\usepackage{tabulary}
\usepackage{amssymb}% http://ctan.org/pkg/amssymb
\usepackage{pifont}% http://ctan.org/pkg/pifont
\usepackage{csquotes}
\usepackage{aurl}
\daurl{hito}{http://hitontology.eu/ontology/}
\newcommand{\cmark}{\ding{51}}%
\newcommand{\xmark}{\ding{55}}%
\usetheme{simple}
\usecolortheme{whiteonblack}

\title{HITO Standards und Kataloge}
%\subtitle{Überblick}
%\author{Konrad Höffner}
\date{2021-03-30}

% Parameters: frame title, image path, (optional) note
\newcommand{\imageslide}[4][]
{
%\newgeometry{margin=0.0cm,top=1em}
\begin{frame}[plain]{~~~~#2}
\vspace{0.2em}
\begin{center}
\centering\includegraphics[width=1.0\textwidth,height=0.95\textheight,keepaspectratio]{#3}
\end{center}
#1
\note{#4}
\end{frame}
%\restoregeometry
}

\begin{document}
\begin{frame}
\titlepage
\end{frame}

\section{Interoperabilitätsstandards}
\imageslide{Updated Interoperability Standards (FJ, MB)}{img/interoperability.pdf}{}
\imageslide{Updated Interoperability Standards (FJ, MB)}{img/interoperability-marked.png}{}

\begin{frame}{Mehr Informationen über Interoperabilitätsstandards}
\begin{itemize}
\item 103 Standards, von Arden Syntax bis YAML
\item \url{http://hitontology.eu/ontology/Interoperability}
\item \url{https://www.snik.eu/graph/?sparql=https://hitontology.eu/sparql&instances}
\item \url{https://github.com/hitontology/ontology/issues/49}
\item Google Drive \enquote{MP131 HITO} $\rightarrow$ \enquote{ontology}\\
      \url{https://docs.google.com/spreadsheets/d/170b3u\_5k3ilvuVnyVFSRY-oN\_tD-sY10qzkIRLVBjKU}
\end{itemize}
\end{frame}

\section{Kataloge}

\begin{frame}{Katalogtypen}
\begin{itemize}
\item \aurl{hito}{FeatureCatalogue} (8)
\item \aurl{hito}{EnterpriseFunctionCatalogue} (3)
\item \aurl{hito}{UserGroupCatalogue} (1, SNOMED CT)
\item \aurl{hito}{OrganizationalUnitCatalogue} (1, SNOMED CT)
\item \aurl{hito}{ApplicationSystemCatalogue} (2)
\item + UnknownXCatalogue (je einer)
\end{itemize}
\end{frame}

\begin{frame}{Explorieren}
\begin{enumerate}
\item \url{http://hitontology.eu/Catalogue} im Browser öffnen
\item Katalogtyp auswählen, \aurl{hito}{FeatureCatalogue}
\item \enquote{Inverse Relations} $\rightarrow$ \enquote{is rdf:type of} $\rightarrow$ Katalog auswählen
\end{enumerate}
\begin{itemize}
\item Kataloge können über \aurl{rdfs}{member} mit einer Instanz von \aurl{hito}{Classification} verbunden sein
\item \aurl{hito}{Classification} gruppiert Katalogue unterschiedlicher Typen mit der selben Quelle\\
Bsp.: \aurl{hito}{WhoDhi1-0} enthält \aurl{hito}{WhoDhiClientFeatureCatalogue}, \aurl{hito}{WhoDhiHealthcareProviderFunctionCatalogue}, \ldots
\end{itemize}
\end{frame}

\begin{frame}{Insgesamt 15 Kataloge (ohne UnknownX)}
\begin{itemize}
\item \aurl{hito}{WhoDhiClientFeatureCatalogue}
\item \aurl{hito}{WhoDhiDataServiceFeatureCatalogue}
\item \aurl{hito}{WhoDhiHealthSystemManagerFeatureCatalogue}
\item \aurl{hito}{WhoDhiHealthcareProviderFeatureCatalogue}
\item \aurl{hito}{WhoDhiSystemCategoryApplicationSystemCatalogue}
\item \aurl{hito}{BbFeatureCatalogue}
\item \aurl{hito}{EhrSfmFeatureCatalogue}
\item \aurl{hito}{JoshiPacsFeatureCatalogue}
\item \aurl{hito}{Dickinson}
\item \aurl{hito}{BbApplicationSystemCatalogue}
\item \aurl{hito}{WhoDhiHealthSystemManagerFunctionCatalogue}
\item \aurl{hito}{WhoDhiHealthcareProviderFunctionCatalogue}
\item \aurl{hito}{BbFunctionCatalogue}
\item \aurl{hito}{SnomedEnvironmentOrganizationalUnitCatalogue}
\item \aurl{hito}{SnomedUserGroupCatalogue}
\end{itemize}
\end{frame}


\end{document}
