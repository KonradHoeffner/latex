\documentclass[14pt,aspectratio=1610]{beamer}
\usepackage[utf8]{inputenc}
\usepackage[ngerman]{babel}
\usetheme{simple}
\usecolortheme{whiteonblack}
\usepackage{array}
\usepackage{tikz}
\usepackage{booktabs}
\usepackage{csquotes}
\usepackage{amssymb}
\newcommand{\xmark}{\ding{55}}%
\usepackage{pifont}
\newcolumntype{H}{>{\setbox0=\hbox\bgroup}c<{\egroup}@{}} % comment out columns
\author{\small Konrad Höffner, Wissenschaftlicher Mitarbeiter, Institut für Medizinische Informatik, Statistik und Epidemiologie. Doktorand, Agile Knowledge Engineering and Semantic Web (AKSW)}
\title{Doktorarbeit:\\\enquote{Question Answering on RDF Data Cubes}}
\subtitle{\small Urspünglicher Erstbetreuer: Prof. Dr. Ing. habil. Dipl.-Math. Klaus-Peter Fähnrich, Institut für Angewandte Informatik, Universität Leipzig\\~\\
Zweitbetreuer: Prof. Dr. Jens Lehmann,\\Institut für Informatik, Universität Bonn}

\begin{document}
\begin{frame}
\titlepage
\end{frame}

%\section{PhD progress \mbox{November 2014--May 2016}}

\begin{frame}[fragile]{Domäne}
\begin{center}
\begin{tikzpicture}
  \scriptsize
  \tikzset{venn circle/.style={circle,minimum width=0.40\textwidth,fill=#1,opacity=0.4}}
  \node[venn circle = red] (SW) at (0,0) {};
  \node[venn circle = blue] (QA) at (60:0.17\textwidth) {};
  \node[venn circle = green] (QB) at (0:0.17\textwidth) {};

  \node[above,rotate=-45,yshift=-0.130\textwidth] at (barycentric cs:SW=1) {Semantic Web /  RDF};
  \node[above,yshift=0.06\textwidth] at (barycentric cs:QA=1) {Question Answering (QA)};
  \node[above,rotate=45,yshift=-0.130\textwidth] at (barycentric cs:QB=1) {Data Cube};
  \node[left,xshift=-0.050\textwidth,yshift=0.050\textwidth,align=center] at (barycentric cs:SW=1/2,QA=1/2) {SQA};
  \node[below,yshift=-0.050\textwidth,align=center] at (barycentric cs:SW=1/2,QB=1/2 ) {RDF\\Data Cube\\(RDC)};
  \node[right,yshift=0.05\textwidth] at (barycentric cs:QA=1/2,QB=1/2 ) {~};
  \node[below,yshift=0.025\textwidth] at (barycentric cs:SW=1/3,QA=1/3,QB=1/3 ){RDCQA};
\end{tikzpicture}
\end{center}
\end{frame}

\begin{frame}[fragile]{Projekte}
\centering
\begin{center}
\begin{tikzpicture}
  \scriptsize
  \tikzset{venn circle/.style={circle,minimum width=0.40\textwidth,fill=#1,opacity=0.4}}
  \node[venn circle = red] (SW) at (0,0) {};
  \node[venn circle = blue] (QA) at (60:0.17\textwidth) {};
  \node[venn circle = green] (QB) at (0:0.17\textwidth) {};

  \node[above,rotate=-45,yshift=-0.130\textwidth] at (barycentric cs:SW=1) {Semantic Web / RDF};
  \node[above,yshift=0.06\textwidth] at (barycentric cs:QA=1) {Question Answering (QA)};
  \node[above,rotate=45,yshift=-0.130\textwidth] at (barycentric cs:QB=1) {Data Cube};
  \node[left,xshift=-0.050\textwidth,yshift=0.050\textwidth,align=center] at (barycentric cs:SW=1/2,QA=1/2) {SQA\\\textbf{2. Survey}};
  \node[below,yshift=-0.050\textwidth,align=center] at (barycentric cs:SW=1/2,QB=1/2 ) {RDF\\Data Cube\\(RDC)\\\textbf{1. LinkedSpending}};
  \node[right,yshift=0.05\textwidth] at (barycentric cs:QA=1/2,QB=1/2 ) {~};
  \node[below,yshift=0.025\textwidth] at (barycentric cs:SW=1/3,QA=1/3,QB=1/3 ){RDCQA \textbf{3. CubeQA}};
\end{tikzpicture}
\end{center}
\end{frame}

\begin{frame}{Kernpublikationen}
\begin{tabular}{lll}
LinkedSpending \cite{linkedspending}	&Semantic Web Journal	&2014\\
SQA Survey \cite{qasurvey}		&Semantic Web Journal	&2017\\
CubeQA Short \cite{cubeqashort}		&SEMANTiCS		&2014\\
CubeQA Long \cite{cubeqa}		&ISWC			&2016\\
\end{tabular}
\\~\\~\\14 andere Publikationen (2 davon Erstauthor).
\end{frame}

\begin{frame}{}
\end{frame}

\begin{frame}[allowframebreaks]{Bibliographie}
\bibliographystyle{unsrt}
\bibliography{konrad}
\end{frame}

\end{document}
