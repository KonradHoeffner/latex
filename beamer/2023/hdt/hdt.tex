\documentclass[14pt,aspectratio=169]{beamer}
\usepackage{ifxetex}
\ifxetex
  \usepackage{fontspec}
\else
  \usepackage[T1]{fontenc}
  \usepackage[utf8]{inputenc}
\fi
\usepackage{pifont}% http://ctan.org/pkg/pifont
\newcommand{\cmark}{\ding{51}}%
%\newcommand{\xmark}{\ding{55}}%

\usepackage[english]{babel}
\uselanguage{english}
\usetheme{simple}
\usecolortheme{whiteonblack}
\usepackage{array}
%\usepackage{aurl}
\usepackage{url}
\usepackage{booktabs}
\usepackage{graphicx}
\usepackage{csquotes}
\usepackage{amssymb}
\usepackage{pifont}
\newcommand{\xmark}{\ding{55}}%
\newcolumntype{H}{>{\setbox0=\hbox\bgroup}c<{\egroup}@{}} % comment out columns
\date{2023-06-28}
\author{
\small
\texorpdfstring{Tim Baccaert\newline{}Your Affiliation\newline{}\url{tim@baccaert.com}}{Tim Baccaert}\\~\\
\texorpdfstring{Konrad Höffner\newline{}Institute for Medical Informatics, Statistics and Epidemiology (IMISE)\newline{}\url{konrad.hoeffner@uni-leipzig.de}}{Konrad Höffner}
}
\title{Header Dictionary Triples (HDT) in Rust}
\subtitle{}

\newcommand{\imageslide}[4][]
{
\begin{frame}[plain]{~~~~#2}
\vspace{0.2em}
\centering\makebox[\linewidth]{\includegraphics[width=1.13\linewidth,height=0.88\textheight,keepaspectratio]{#3}}
\\#1
\note{#4}
\end{frame}
}

\newcommand\pro{\item[$+$]}
\newcommand\con{\item[$-$]}

\begin{document}

\begin{frame}
\titlepage
\end{frame}

% if Wouter Beek says there are only HDT-experts attending, then intro may be shortened
\section{Introduction}

\begin{frame}{Problem}
\enquote{
\ldots{}what the Semantic Web field most needs, at this stage, is consolidation\ldots
For \textbf{academics}, there is often limited incentive to develop and maintain \textbf{stable, easy-to-use software}\ldots
Consolidation of sorts is already happening in \textbf{industry}\ldots
Technical details\ldots{} are however \textbf{usually not shared}, presumably to protect the own competitive edge.
}
\\~\\
Pascal Hitzler: \emph{A Review of the Semantic Web Field}. 2021
\end{frame}

% assumption: everyone knows what RDF is, maybe 1-2 don't know what HDT is
% can skip it if everyone knows
\begin{frame}{Consuming and Querying RDF}
\begin{itemize}
\item text-based serialization formats: N-Triples, RDF Turtle or RDF/XML
\item large file sizes
\item high compression ratios with general-purpose algorithms
\item download, uncompress, import into SPARQL endpoint, query
\end{itemize}
\end{frame}

\imageslide{Example: RDF Browser with SPARQL}{img/architecture.pdf}{}

% can also comment this out and just say it while showing architecture slide
\begin{frame}{SPARQL Endpoints}
\begin{itemize}
\pro Linked Open Data best practice
\pro standard RDF query language
\pro reuse of standard tools
\pro update graphs with SPARUL
\pause
\con overhead on small graphs
\con timeouts
\con downtimes
\con separate component: failure point, hard to integrate
\end{itemize}
\end{frame}

%\imageslide{Example: RDF Browser with RDF library}{img/architecture-simple.pdf}{}

\imageslide[\footnotesize Source: \emph{Martínez-Prieto et al: Exchange and consumption of huge RDF data.} 2012.]{HDT Bitmap Triples}{img/bt.png}{}

\begin{frame}[plain]{HDT Focussed on Querying}
\centering
\makebox[\linewidth]{\includegraphics[height=0.6\textheight]{img/hdt-foq.png}}\\
\footnotesize Source: \emph{Martínez-Prieto et al: Exchange and consumption of huge RDF data.} 2012.
\end{frame}

\begin{frame}[fragile]{RDF Libraries: ?PO, 10M triples}
\centering
\small
\begin{tabular}{lrrr}
\toprule
Library & Memory in MB & Load Time in ms & Query Time in ms \\
\midrule
hdt\_cpp & \textbf{112} & 1985 & 362 \\
sophia\_hdt & 263 & 930 & 355 \\
hdt\_rs & 264 & \textbf{912} & 315 \\
hdt\_java (DS) & 738 & 3170 & \textbf{214} \\
hdt\_java (String) & 785 & 3476 & \textbf{321} \\
\midrule
sophia\_lg & \textbf{834} & \textbf{11656} & 85 \\
sophia & 1371 & 15990 & \textbf{20} \\
jena (java) & 5352 & 40400 & 159 \\
n3js (js) & 12404 & 100820 & 654 \\
rdflib (python) & 14481 & 182002 & 940 \\
librdf (c) & -- & -- & -- \\
\bottomrule
\end{tabular}
\end{frame}

%\imageslide{Simplified Architecture}{img/architecture-simple.pdf}{}

\begin{frame}[plain]{RDF Libraries}
\centering
\makebox[\linewidth]{\includegraphics[height=0.7\textheight]{img/libraries.png}}\\
\scriptsize Source: \emph{Höffner \& Baccaert:\\hdt-rs: A Rust library for the Header Dictionary Triples binary RDF compression format. 2023.}
\end{frame}

\begin{frame}{Rust---Ideal fit for the Semantic Web}
\begin{itemize}
\pro fast
\pro low memory
\pro static guarantees
\pro modern features
\pro builtin tooling: formatting, linting, testing, benchmarking
\pro concise and fun
\pro fast evolution
\pause
\con complex: traits, borrow checker
\con new, less libraries
\con graph-like structures are difficult
\end{itemize}
\end{frame}


\begin{frame}{Conclusions}
\begin{itemize}
%\item long-term availability of knowledge graphs matters
\item Rust great fit for stable and performant Semantic Web tooling
\item 
\end{itemize}
\end{frame}
\fi

\begin{frame}[fragile]{Thanks! Questions?}
\centering
Issues and PRs welcome at \url{https://github.com/KonradHoeffner/hdt}\\
~\\
\url{tim@baccaert.com}\\
\url{konrad.hoeffner@uni-leipzig.de}\\
\end{frame}

\end{document}
