\documentclass[14pt,aspectratio=169]{beamer}
\usepackage{ifxetex}
\ifxetex
  \usepackage{fontspec}
\else
  \usepackage[T1]{fontenc}
  \usepackage[utf8]{inputenc}
\fi
\usepackage{pifont}% http://ctan.org/pkg/pifont
\newcommand{\cmark}{\ding{51}}%
%\newcommand{\xmark}{\ding{55}}%

\usepackage[english]{babel}
\uselanguage{english}
\usetheme{simple}
\usecolortheme{whiteonblack}
\usepackage{array}
%\usepackage{aurl}
\usepackage{url}
\usepackage{booktabs}
\usepackage{graphicx}
\usepackage{csquotes}
\usepackage{amssymb}
\usepackage{pifont}
\newcommand{\xmark}{\ding{55}}%
\newcolumntype{H}{>{\setbox0=\hbox\bgroup}c<{\egroup}@{}} % comment out columns
\date{2023-06-28}
\author{
\small
\texorpdfstring{Tim Baccaert\newline{}Software Languages Lab, Vrije Universiteit Brussel\newline{}\url{tim.baccaert@vub.be}}{Tim Baccaert}\\~\\
\texorpdfstring{Konrad Höffner\newline{}Institute for Medical Informatics, Statistics and Epidemiology (IMISE)\newline{}\url{konrad.hoeffner@uni-leipzig.de}}{Konrad Höffner}
}
\title{Header Dictionary Triples (HDT) in Rust}
\subtitle{}

\newcommand{\imageslide}[4][]
{
\begin{frame}[plain]{~~~~#2}
\vspace{0.2em}
\centering\makebox[\linewidth]{\includegraphics[width=1.13\linewidth,height=0.8\textheight,keepaspectratio]{#3}}
\\#1
\note{#4}
\end{frame}
}

\newcommand\pro{\item[$+$]}
\newcommand\con{\item[$-$]}

\begin{document}

\begin{frame}
\titlepage
\end{frame}

% if Wouter Beek says there are only HDT-experts attending, then intro may be shortened
\section{Introduction}

% cut to save time
\iffalse
\begin{frame}{Problem}
\enquote{
\ldots{}what the Semantic Web field most needs, at this stage, is consolidation\ldots
For \textbf{academics}, there is often limited incentive to develop and maintain \textbf{stable, easy-to-use software}\ldots
Consolidation of sorts is already happening in \textbf{industry}\ldots
Technical details\ldots{} are however \textbf{usually not shared}, presumably to protect the own competitive edge.
}
\\~\\
Pascal Hitzler: \emph{A Review of the Semantic Web Field}. 2021
\end{frame}
\fi

% assumption: everyone knows what RDF is, maybe 1-2 don't know what HDT is
% can skip it if everyone knows
\begin{frame}{Consuming and Querying RDF}
\begin{itemize}
\item default RDF formats are text-based
\item large KB $\rightarrow$ large file size
\item high compression ratios with general-purpose algorithms
\item cannot query directly
\item download, import into SPARQL endpoint, query
\end{itemize}
\end{frame}

\imageslide[best practice to disseminate a knowledge base\\+ reuse of standard tools -overhead -failure point -integration]
{SPARQL Endpoint as central component}{img/architecture.pdf}{}

% can also comment this out and just say it while showing architecture slide
\iffalse
\begin{frame}{SPARQL Endpoints}
\begin{itemize}
\pro Linked Open Data best practice
\pro standard RDF query language
\pro reuse of standard tools
\pro update graphs with SPARUL
\pause
\con overhead on small graphs
\con timeouts
\con downtimes
\con separate component: failure point, hard to integrate
\end{itemize}
\end{frame}
\fi

\imageslide{Inversion: query RDF inside the application}{img/architecture-simple.pdf}{}

\begin{frame}[plain]{RDF Libraries}
\centering
\makebox[\linewidth]{\includegraphics[height=0.7\textheight]{img/libraries.png}}\\
\scriptsize Source: \emph{Höffner \& Baccaert:\\hdt-rs: A Rust library for the Header Dictionary Triples binary RDF compression format. 2023.}
\end{frame}

\begin{frame}[fragile]{RDF Libraries: ?PO, 10M triples}
\centering
\small
\begin{tabular}{lrrr}
\toprule
Library & Memory in MB & Load Time in ms & Query Time in ms \\
\midrule
hdt\_cpp & \textbf{112} & 1985 & 362 \\
sophia\_hdt & 263 & 930 & 355 \\
hdt\_rs & 264 & \textbf{912} & 315 \\
hdt\_java (DS) & 738 & 3170 & \textbf{214} \\
hdt\_java (String) & 785 & 3476 & \textbf{321} \\
\midrule
sophia\_lg & \textbf{834} & \textbf{11656} & 85 \\
sophia & 1371 & 15990 & \textbf{20} \\
jena (java) & 5352 & 40400 & 159 \\
n3js (js) & 12404 & 100820 & 654 \\
rdflib (python) & 14481 & 182002 & 940 \\
librdf (c) & -- & -- & -- \\
\bottomrule
\end{tabular}
\end{frame}

\imageslide[\footnotesize Source: \emph{Martínez-Prieto et al: Exchange and consumption of huge RDF data.} 2012.]{HDT: RDF compression allowing fast queries}{img/bt.png}{}

\begin{frame}[plain]{HDT Focussed on Querying}
\centering
\makebox[\linewidth]{\includegraphics[height=0.6\textheight]{img/hdt-foq.png}}\\
\footnotesize Source: \emph{Martínez-Prieto et al: Exchange and consumption of huge RDF data.} 2012.
\end{frame}

\begin{frame}{Rust---Great fit for Semantic Web tooling}
\begin{itemize}
\pro fast
\pro low memory
\pro static guarantees
\pro Cargo tooling
\pro fast evolution
\pro concise and fun
\pause
\con complex: traits, borrow checker
\con new, less libraries
\con graph-like structures are difficult
\end{itemize}
\end{frame}

\section{HDT\_rs: some history, and the core of the library (Tim)}

\begin{frame}{A brief history}
    \begin{itemize}
        \item originally for my master's thesis
            \begin{itemize}
                \item Datalog variant for eventually consistent RDF querying
                \item needed huge volumes of benchmark data.
                \item hdt files are compressed $\implies$ easier to share data dumps of
            \end{itemize}
        \item time constraints: decided to focus on my proofs only
        \item open-sourced what I had, which was incomplete
        \item Konrad picked it up and made it actually usable for real
            \begin{itemize}
                \item really deserves the vast majority of the credit :)
            \end{itemize}
    \end{itemize}
\end{frame}

\begin{frame}{Basic architecture}
    \begin{itemize}
        \item resembles Java and C++ versions of hdt
        \item architecture has 4 major components
            \begin{itemize}
                \item \textbf{api}: how do users interact with the library
                \item \textbf{model}: in-memory representation of triples
                \item \textbf{containers}: model of compressed on-disk data (PFC strings)
                \item \textbf{dictionaries}: mapping between triple bitmaps and their compressed
                    components
            \end{itemize}
    \end{itemize}
\end{frame}

\begin{frame}[fragile]{Read path: initialisation time}
    \textbf{Goal:} compact in-memory representation of triples
    \begin{verbatim}
        // [usize, usize, usize],
        [1, 2, 2],
        [3, 5, 7],
        ...
    \end{verbatim}
    \begin{enumerate}
        \item Read the metadata (e.g., triple order, version, ...)
        \item read the dictionary: mapping between identifiers and actual values
        \item read the triple bitmap: id-based representation of triples
        \item build index over bitmap
    \end{enumerate}
\end{frame}

\begin{frame}{Read path: iteration}
    \begin{enumerate}
        \item we get a triple pattern \texttt{[1, 2, 3]} in \texttt{SPO} order from the index
        \item lookup 1, 2, and 3 in the dictionary
            \begin{itemize}
                \item strings are stored with plain front coding
                \item basic idea: exploit that URIs always start with \texttt{https://www.foo}
                \item store the prefix only once, keep only what is different otherwise
            \end{itemize}
        \item return the full, glorious, in-memory representation of the triple
    \end{enumerate}
\end{frame}

\section{Optimization}

\imageslide[wavelet matrix generated late]{Peak RAM usage: reorder loading I}{img/matrix-last.png}{}
\imageslide[wavelet matrix generated early]{Peak RAM usage: reorder loading II}{img/matrix-first.png}{}

\imageslide[Parallelize loading]{Profile and optimize CPU performance}{img/parallel.png}{}

\section{Challenge: How to return triples?}

\begin{frame}{Which collection for triples?}
\begin{itemize}
\item an RDF graph is a set of triples
\item a triple pattern matches an RDF subgraph
\item return a \texttt{Set<Triple>}?
\pause
\con we may only care about a part of the results
\con all processing needs to happen at the start
%\item in an uncompressed graph, \texttt{Set<\&Triple>} could make sense
\con memory needs to be reserved all at once
\item $\rightarrow$ \texttt{Iterator<Triple>}
\end{itemize}
\end{frame}

\begin{frame}[fragile]{HDT Rust API}
\small
\begin{verbatim}
pub fn triples_with_pattern<'a>(
    &'a self,
    sp: Option<&'a str>,
    pp: Option<&'a str>,
    op: Option<&'a str>)

  -> Box<dyn Iterator<Item = StringTriple> + '_> {
...
\end{verbatim}
\end{frame}

\begin{frame}{How to represent a single triple?}
\centering
\makebox[\linewidth]{\includegraphics[width=\textwidth]{img/bt.png}}\\
\begin{itemize}
\item Triple section contains triples of integer IDs
\item official HDT docs: unsigned 32 bit
\item HDT Rust uses usize (64 bit on 64 bit platform) but could be changed to u32
\item HDT Java uses long (signed 64 bit)
\item Translate IDs to strings using dictionary
\item $\texttt{Triple} = \texttt{(String, String, String)}$?
\end{itemize}
\end{frame}

\begin{frame}{String in Java}
\begin{itemize}
\pro immutable
\pro non-primitive, so actually a reference
\pro duplicates (e.g SP? pattern) don't waste space and performance
\pro garbage collected when not referenced anymore
\end{itemize}
\end{frame}

\begin{frame}[fragile]{HDT Java}
\small
\begin{verbatim}
public class TripleString {
    private CharSequence subject;
    private CharSequence predicate;
    private CharSequence object;
}

//  CharSequence impl that uses only one byte per char
public class CompactString
   implements CharSequence, { getDelayed() ... }
\end{verbatim}
\end{frame}

\begin{frame}[fragile]{HDT Java II}
\small
\begin{verbatim}
public final class DelayedString implements CharSequence {
  private CharSequence str;

  private void ensure() {
    if(!(str instanceof String)) {str = str.toString();}
  }

  @Override
  public String toString() {
    ensure();
    return str.toString();
  }
...
\end{verbatim}
\end{frame}

\begin{frame}[fragile]{HDT Java III}
\small
\begin{verbatim}
public class DictionaryTranslateIteratorBuffer
    implements IteratorTripleString {
Map<Long,CharSequence> mapSubject, mapPredicate, mapObject;
...
\end{verbatim}
\end{frame}

\begin{frame}[fragile]{String in Rust}
\begin{itemize}
\item pointer to the heap
\con may grow, capacity field uses one extra usize (alternative: \texttt{Box<str>})
\con ownership of the contents\\$\rightarrow$ waste space on duplication
\end{itemize}
\begin{verbatim}
<http://looongsubject.com> <http://ex.com/label> "Example Resource".
<http://loongsubject.com> <http://ex.com/comment> "wasted space".
\end{verbatim}
\end{frame}

\begin{frame}{Pointer to String?}
\begin{itemize}
\item Rust has raw pointers like \texttt{*const String}, but they...
\con are not guaranteed to point to valid memory
\con are not automatically cleaned up
\con don't move ownership
\con dereferencing requires \emph{unsafe} Rust
\item HDT Rust only uses \emph{safe} Rust
\end{itemize}
\end{frame}

\begin{frame}[fragile]{Reference to String}
\small
\begin{verbatim}
let string = String::from("Hello, world!");
// Convert the String to a string slice
let s: &str = string.as_str();

// Compile error!
// Cannot return a reference to a value created in a function.
return s;
\end{verbatim}
\begin{itemize}
\item Cannot return reference within HDT object because resources only exist in compressed form.
\end{itemize}
\end{frame}

\begin{frame}[fragile]{Maybe Owned String}
\begin{itemize}
\item \url{https://crates.io/crates/mownstr}
\item String that may be owned or borrowed
\item used by HDT Rust in the past
\pro SP? pattern query: return reference to subject and predicate parameters, create new string for each object
\con does not work e.g. for ??? pattern
\end{itemize}
\small
\begin{verbatim}
pub fn triples_with_pattern<'a>(&'a self,
    sp: Option<&'a str>,
    pp: Option<&'a str>,
    op: Option<&'a str>)

  -> Box<dyn Iterator<Item = StringTriple> + '_> { ...
...
\end{verbatim}
\end{frame}

\begin{frame}[fragile]{\texttt{Arc<str>}}
\begin{itemize}
\item atomic (thread safe) reference counting smart pointer
\pro safe space on ??? pattern
\con synchronization overhead $\rightarrow$ not optimal for patterns like SP?
\item best compromise right now
\end{itemize}
\end{frame}

\begin{frame}[fragile]{One pattern query method vs many?}
\footnotesize
\begin{verbatim}
pub fn all_triples ...
pub fn triples_with_s ...
pub fn triples_with_p ...
pub fn triples_with_o ...
pub fn triples_with_sp( ...
pub fn triples_with_so( ...
pub fn triples_with_po( ...
pub fn triples_with_spo( ...
\end{verbatim}
\begin{itemize}
\pro best performance
\con different return types
\con code duplication
\con complicated for library users
\end{itemize}
\end{frame}

\iffalse
\begin{frame}{Conclusions}
\begin{itemize}
\item Rust great fit for stable and performant Semantic Web tooling
\end{itemize}
\end{frame}
\fi

\begin{frame}[fragile]{Thanks! Questions?}
\centering
HDT Rust available at \url{https://crates.io/crates/hdt}\\
Issues and PRs welcome at \url{https://github.com/KonradHoeffner/hdt}\\
~\\
\url{tim.baccaert@vub.be}\\
\url{konrad.hoeffner@uni-leipzig.de}\\
\end{frame}

\end{document}
