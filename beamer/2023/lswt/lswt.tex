\documentclass[14pt,aspectratio=169]{beamer}
\usepackage{ifxetex}
\ifxetex
  \usepackage{fontspec}
\else
  \usepackage[T1]{fontenc}
  \usepackage[utf8]{inputenc}
\fi
\usepackage{pifont}% http://ctan.org/pkg/pifont
\newcommand{\cmark}{\ding{51}}%
%\newcommand{\xmark}{\ding{55}}%

\usepackage[english]{babel}
\uselanguage{english}
\usetheme{simple}
\usecolortheme{whiteonblack}
\usepackage{array}
\usepackage{aurl}
\daurl{meta}{http://www.snik.eu/ontology/meta/}
\daurl{ob}{http://www.snik.eu/ontology/ob/}
\daurl{bb}{http://www.snik.eu/ontology/bb/}
\usepackage{url}
\usepackage{graphicx}
\usepackage{csquotes}
\usepackage{amssymb}
\usepackage{pifont}
\newcommand{\xmark}{\ding{55}}%
\newcolumntype{H}{>{\setbox0=\hbox\bgroup}c<{\egroup}@{}} % comment out columns
\date{2023-06-28}
\author{\texorpdfstring{Konrad Höffner\newline{}Institute for Medical Informatics, Statistics and Epidemiology (IMISE)\newline{}\url{konrad.hoeffner@uni-leipzig.de}}{Konrad Höffner}}
\title{RickView: Lightweight Standalone Knowledge Graph Browsing Powered by Rust}
\subtitle{}

\newcommand{\imageslide}[4][]
{
\begin{frame}[plain]{~~~~#2}
\vspace{0.2em}
\centering\makebox[\linewidth]{\includegraphics[width=1.13\linewidth,height=0.88\textheight,keepaspectratio]{#3}}
\\#1
\note{#4}
\end{frame}
}

\begin{document}

\begin{frame}
\titlepage
\end{frame}

\iffalse
\begin{frame}{Problem}
\enquote{
\ldots{}what the Semantic Web field most needs, at this stage, is consolidation\ldots
For academics, there is often limited incentive to develop and maintain stable, easy-to-use software\ldots
Consolidation of sorts is already happening in industry\ldots
Technical details\ldots{} are however usually not shared, presumably to protect the own competitive edge.
}
Source: P. Hitzler: \emph{A Review of the Semantic Web Field}. 2021
\end{frame}
\fi

\begin{frame}{Problem}
\begin{itemize}
\item best practice to have RDF browser for every KG 
\item after a while, you have a lot of them
\item deployments break, servers shut down, lack of responsibility and resources
%\item quick prototyping for new projects
%\item lack of stable and easy-to-use open source Semantic Web tooling
%\item new knowledge graph in every project
%\item RDF browser needed as intuitive entry point
%\item 
%\item funding and responsibility limited to project duration 
%\item no operational management for indefinite deployment
%\pause
%\item downtime and abandonment of deployment harm knowledge graph value significantly
%\item time-consuming maintenance of past projects hinders current progress
\end{itemize}
~\\
\centering
\emph{downtime and abandonment}\\
vs\\
\emph{time and money}
\end{frame}

\begin{frame}{Goal}
\begin{itemize}
\item generic RDF browser: Pubby-class URI dereferencing tool
\item easy to use
\item good design
\item lightweight
\item standalone%: less failure points
\item easy configuration 
\item high performance
%\pause
%\item intuitive
%\item good design
%\item robust
%\pause
%\item efficient development
\end{itemize}
\end{frame}

\iffalse
\begin{frame}{Linked Open Data Dissemination Best Practices}
\begin{itemize}
\item RDF Dump
\item SPARQL Endpoint
\item RDF Browser
\pause
\item Collaborative Editing
\item Blogs, Videos, Social Media, \ldots
\end{itemize}
\end{frame}
\fi

\iffalse
\begin{frame}{Rust---Ideal fit for the Semantic Web}
\begin{itemize}
\item fast
\item low memory
\item safe
\item modern features and tooling
\pause
\item however not that many libraries yet
\end{itemize}
\end{frame}
\fi

\imageslide{Common Architecture}{img/architecture.pdf}{}

\begin{frame}[plain]{~~~~Example: LodView}
%\begin{itemize}
%\item intuitive \cmark
%\item good design \cmark
%\item robust
%\item lightweight: many instances on small servers
%\item standalone: less failure points
%\item easy setup and configuration
%\item high performance on client and server
%\pause
%\end{itemize}

\vspace{0.2em}
\centering\makebox[\linewidth]{\includegraphics[width=1.13\linewidth,height=0.7\textheight,keepaspectratio]{img/lodview-screenshot.png}}
\footnotesize
easy to use \cmark{}~~good design \cmark{}~~lightweight ?
%
%lightweight \xmark{}~~standalone \xmark{}~~performant \xmark{}~~efficient development \xmark{}~~adaptable $\approx$ \\
\end{frame}

\imageslide[KB: SNIK, 2.1MB RDF/Turtle.\\LodView server: 533 MB RAM + 210 MB Virtuoso]{LodView---Network}{img/lodview-nocache.png}{}
\imageslide[KB: SNIK, 2.1MB RDF/Turtle.\\LodView server: 533 MB RAM + 210 MB Virtuoso]{LodView---Cached}{img/lodview-cache.png}{}

\imageslide{RickView}{img/rickview-screenshot1.png}{}
\imageslide{RickView}{img/rickview-screenshot2.png}{}

\imageslide{Simplified Architecture}{img/architecture-simple.pdf}{}

\begin{frame}[plain]{RDF Libraries}
\centering
\makebox[\linewidth]{\includegraphics[height=0.7\textheight]{img/libraries.png}}\\
\scriptsize Source: \emph{Höffner \& Baccaert:\\hdt-rs: A Rust library for the Header Dictionary Triples binary RDF compression format. 2023.}
\end{frame}

\begin{frame}{Triple Patterns}
\begin{itemize}
\item visualize a resource using all triples where it occurs in subject or object position
\item expressivity of SPARQL not required
\item \emph{direct relations}---S?? pattern
\item \emph{inverse relations}---??O pattern
\end{itemize}
\end{frame}

%\imageslide{RDF Libraries}{img/libraries.png}{}
\imageslide{RickView Architecture}{img/architecture-rickview.pdf}{}

%\imageslide{LodView}{img/lodview-screenshot.png}{}
\begin{frame}[plain]{RickView---Network}
\centering
\makebox[\linewidth]{\includegraphics[width=1.13\linewidth]{img/nocache.png}}\\
%\hspace{2em}
\pause
Cached\\
\centering\makebox[\linewidth]{\includegraphics[width=1.13\linewidth]{img/cache.png}}
KB: SNIK, 2.1MB RDF/Turtle.\\
RickView 0.2.2 server: 25 MB RAM\\
binary size 11 MB, Alpine Docker image 18 MB
\end{frame}

\imageslide[\footnotesize Source: \emph{Martínez-Prieto et al: Exchange and consumption of huge RDF data.} 2012.]{Direct Relations---HDT Bitmap Triples}{img/bt.png}{}

\begin{frame}[plain]{Inverse Relations---HDT FoQ}
\centering
\makebox[\linewidth]{\includegraphics[height=0.6\textheight]{img/hdt-foq.png}}\\
\footnotesize Source: \emph{Martínez-Prieto et al: Exchange and consumption of huge RDF data.} 2012.
\end{frame}

\begin{frame}[plain]{RickView LinkedSpending---Network}
\centering\makebox[\linewidth]{\includegraphics[width=1.13\linewidth]{img/linkedspending-nocache.png}}\\
\hspace{2em}
\centering\makebox[\linewidth]{\includegraphics[width=1.13\linewidth]{img/linkedspending-cache.png}}
KB: LinkedSpending 2015, 28 GB RDF/N-Triples, 1.4 GB HDT\\
RickView 0.2.2 server: 1.9 GB RAM
\end{frame}

\iffalse
\begin{frame}{Standalone}
\begin{itemize}
\item single binary $\approx 11$ MB, Alpine Docker image $\approx 18$ MB
\item includes web server
\item resources compiled into the binary
\item loads knowledge base from file or URL at start
\end{itemize}
\end{frame}

\begin{frame}{Use Case: HITO\footnote{v1.1 \url{https://hitontology.eu/ontology/}}}
\begin{itemize}
\item RickView 0.2.1: 25 MB RAM
\item LodView 1.2.1 SNIK fork: 533 MB RAM + 210 MB Virtuoso
%\item Docker ulimit extension required
\end{itemize}
\end{frame}

\begin{frame}{Use Case: Linked Spending\footnote{\url{https://linkedspending.aksw.org/}}}
\begin{itemize}
\item v2014-3 file size $\approx 25$ GB N-Triples, $\approx 113$ million triples
\item original setup: extremely slow page load, crashes
\item too large for RickView, $> 32$ GB server RAM usage
\end{itemize}
\end{frame}

\begin{frame}{Use Case: Linked Spending \& HDT}
\begin{itemize}
\item Header Dictionary Triples (HDT)---compressed RDF with efficient triple pattern queries
\item Rust implementation\footnote{\url{https://github.com/konradhoeffner/hdt}} with Tim Baccaert
\item HDT Adapter for Sophia, used by RickView
\item RickView with LinkedSpending v2015 and HDT:\\$\approx 1.9$GB RAM,  $<2$ms page generation time
\end{itemize}
\end{frame}

\begin{frame}{Dependencies}
\begin{tabular}{ll}
RDF Library		&Sophia\\% \cite{sophia}
HDT Library		&hdt-rs\\
Web Framework	&Actix Web\\
Template Engine	&TinyTemplate\\
\end{tabular}
\end{frame}

\begin{frame}{Conclusions}
\begin{itemize}
%\item long-term availability of knowledge graphs matters
\item Rust great fit for stable and performant Semantic Web tooling
\item 
\end{itemize}
\end{frame}

| Library         | Memory in MB | Load Time in ms | Query Time in ms |
|:----------------|-------------:|----------------:|-----------------:|
| hdt_cpp         |      **112** |            1985 |              362 |
| sophia_hdt      |          263 |             930 |              355 |
| hdt_rs          |          264 |         **912** |              315 |
| hdt_java (DelayedString) | 738 |            3170 |          **214** |
| hdt_java (String)        | 785 |            3476 |          **321** |
| sophia_lg       |      **834** |       **11656** |               85 |
| sophia          |         1371 |           15990 |           **20** |
| jena (java)     |         5352 |           40400 |              159 |
| n3js (js)       |        12404 |          100820 |              654 |
| rdflib (python) |        14481 |          182002 |              940 |
| librdf (c)      |           -- |              -- |               -- |
\fi

\begin{frame}[fragile]{Thanks! Questions?}
~\\
Issues and PRs welcome at \url{https://github.com/KonradHoeffner/rickview}\\
~\\
Docker Compose example:
\scriptsize
\begin{verbatim}
services:
  rickview:
    image: ghcr.io/konradhoeffner/rickview
    environment:
      - RICKVIEW_KB_FILE=https://raw.githubusercontent.com/hitontology/ontology/dist/all.ttl
      - RICKVIEW_NAMESPACE=http://hitontology.eu/ontology/
      - RICKVIEW_TITLE=HITO
      - RICKVIEW_SUBTITLE=Health IT Ontology
      - RICKVIEW_EXAMPLES=Study SoftwareProduct
    ports:
      - "127.0.0.1:8080:8080"
\end{verbatim}
\end{frame}

\end{document}
