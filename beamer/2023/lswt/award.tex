\documentclass[14pt,aspectratio=169]{beamer}
\usepackage{ifxetex}
\ifxetex
  \usepackage{fontspec}
\else
  \usepackage[T1]{fontenc}
  \usepackage[utf8]{inputenc}
\fi
\usepackage{pifont}% http://ctan.org/pkg/pifont
\newcommand{\cmark}{\ding{51}}%
%\newcommand{\xmark}{\ding{55}}%

\usepackage[english]{babel}
\uselanguage{english}
\usetheme{simple}
\usecolortheme{whiteonblack}
\usepackage{array}
\usepackage{aurl}
\daurl{meta}{http://www.snik.eu/ontology/meta/}
\daurl{ob}{http://www.snik.eu/ontology/ob/}
\daurl{bb}{http://www.snik.eu/ontology/bb/}
\usepackage{url}
\usepackage{graphicx}
\usepackage{csquotes}
\usepackage{amssymb}
\usepackage{pifont}
\newcommand{\xmark}{\ding{55}}%
\newcolumntype{H}{>{\setbox0=\hbox\bgroup}c<{\egroup}@{}} % comment out columns
\date{2024-09-17}
\author{\texorpdfstring{Konrad Höffner\newline{}Institut für Medizinische Informatik, Statistik und Epidemiologie (IMISE)\newline{}\url{konrad.hoeffner@uni-leipzig.de}}{Konrad Höffner}}
\title{RickView: Robuster, leichtgewichtiger, unabhängiger und performanter Wissensgraphen-Browser}
\subtitle{}

\newcommand{\imageslide}[4][]
{
\begin{frame}[plain]{~~~~#2}
\vspace{0.2em}
\centering\makebox[\linewidth]{\includegraphics[width=1.13\linewidth,height=0.88\textheight,keepaspectratio]{#3}}
\\#1
\note{#4}
\end{frame}
}

\begin{document}

\begin{frame}
\titlepage
\end{frame}

\iffalse
\begin{frame}{Problem}
\enquote{
\ldots{}what the Semantic Web field most needs, at this stage, is consolidation\ldots
For academics, there is often limited incentive to develop and maintain stable, easy-to-use software\ldots
Consolidation of sorts is already happening in industry\ldots
Technical details\ldots{} are however usually not shared, presumably to protect the own competitive edge.
}
Source: P. Hitzler: \emph{A Review of the Semantic Web Field}. 2021
\end{frame}
\fi

\begin{frame}{Problem}
\begin{itemize}
\item best practice to have RDF browser for every KG 
\item after a while, you have a lot of them
\item deployments break, servers shut down, lack of responsibility and resources
%\item quick prototyping for new projects
%\item lack of stable and easy-to-use open source Semantic Web tooling
%\item new knowledge graph in every project
%\item RDF browser needed as intuitive entry point
%\item 
%\item funding and responsibility limited to project duration 
%\item no operational management for indefinite deployment
%\pause
%\item downtime and abandonment of deployment harm knowledge graph value significantly
%\item time-consuming maintenance of past projects hinders current progress
\end{itemize}
~\\
\centering
\emph{downtime and abandonment}\\
vs\\
\emph{time and money}
\end{frame}

\begin{frame}{Goal}
\begin{itemize}
\item generic RDF browser: Pubby-class URI dereferencing tool
\item easy to use
\item good design
\item lightweight
\item standalone%: less failure points
\item easy configuration 
\item high performance
%\pause
%\item intuitive
%\item good design
%\item robust
%\pause
%\item efficient development
\end{itemize}
\end{frame}

\iffalse
\begin{frame}{Linked Open Data Dissemination Best Practices}
\begin{itemize}
\item RDF Dump
\item SPARQL Endpoint
\item RDF Browser
\pause
\item Collaborative Editing
\item Blogs, Videos, Social Media, \ldots
\end{itemize}
\end{frame}
\fi

\iffalse
\begin{frame}{Rust---Ideal fit for the Semantic Web}
\begin{itemize}
\item fast
\item low memory
\item safe
\item modern features and tooling
\pause
\item however not that many libraries yet
\end{itemize}
\end{frame}
\fi

\imageslide{Common Architecture}{img/architecture.pdf}{}

\begin{frame}[plain]{~~~~Example: LodView}
%\begin{itemize}
%\item intuitive \cmark
%\item good design \cmark
%\item robust
%\item lightweight: many instances on small servers
%\item standalone: less failure points
%\item easy setup and configuration
%\item high performance on client and server
%\pause
%\end{itemize}

\vspace{0.2em}
\centering\makebox[\linewidth]{\includegraphics[width=1.13\linewidth,height=0.7\textheight,keepaspectratio]{img/lodview-screenshot.png}}
\footnotesize
easy to use \cmark{}~~good design \cmark{}~~lightweight ?
%
%lightweight \xmark{}~~standalone \xmark{}~~performant \xmark{}~~efficient development \xmark{}~~adaptable $\approx$ \\
\end{frame}


\imageslide{RickView}{img/rickview-screenshot1.png}{}
\imageslide{RickView}{img/rickview-screenshot2.png}{}

\imageslide{Simplified Architecture}{img/architecture-simple.pdf}{}

\begin{frame}[plain]{RDF Libraries}
\centering
\makebox[\linewidth]{\includegraphics[height=0.7\textheight]{img/libraries.png}}\\
\scriptsize Source: \emph{Höffner \& Baccaert:\\hdt-rs: A Rust library for the Header Dictionary Triples binary RDF compression format. 2023.}
\end{frame}

\begin{frame}{Triple Patterns}
\begin{itemize}
\item visualize a resource using all triples where it occurs in subject or object position
\item expressivity of SPARQL not required
\item \emph{direct relations}---S?? pattern
\item \emph{inverse relations}---??O pattern
\end{itemize}
\end{frame}

%\imageslide{RDF Libraries}{img/libraries.png}{}
\imageslide{RickView Architecture}{img/architecture-rickview.pdf}{}

\begin{frame}[fragile]{Thanks! Questions?}
~\\
Issues and PRs welcome at \url{https://github.com/KonradHoeffner/rickview}\\
~\\
Docker Compose example:
\scriptsize
\begin{verbatim}
services:
  rickview:
    image: ghcr.io/konradhoeffner/rickview
    environment:
      - RICKVIEW_KB_FILE=https://raw.githubusercontent.com/hitontology/ontology/dist/all.ttl
      - RICKVIEW_NAMESPACE=http://hitontology.eu/ontology/
      - RICKVIEW_TITLE=HITO
      - RICKVIEW_SUBTITLE=Health IT Ontology
      - RICKVIEW_EXAMPLES=Study SoftwareProduct
    ports:
      - "127.0.0.1:8080:8080"
\end{verbatim}
\end{frame}

\end{document}
